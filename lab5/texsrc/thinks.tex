\section*{Вывод}
В ходе выполнения лабораторной работы были получены и закреплены практические навыки диагностики работы программного обеспечения с помощью системной трассировки (\texttt{strace}).

Был проведен анализ системных вызовов ядра Linux для четырех реализованных лабораторных работ, которые демонстрируют ключевые механизмы операционных систем:
\begin{enumerate}
    \item Взаимодействие процессов через неименованный канал (Pipe), подтвержденное вызовом \texttt{pipe2} и передачей данных через \texttt{read}/\texttt{write}.
    \item Использование многопоточности для параллельных вычислений, подтвержденное многократными вызовами \texttt{clone3} с флагами \texttt{CLONE\_THREAD}.
    \item Межпроцессное взаимодействие через разделяемую память (\texttt{mmap} с \texttt{MAP\_SHARED}) с использованием сигналов
    (\texttt{kill}/\texttt{pause}) для синхронизации.
    \item Динамическая загрузка и выгрузка библиотек, подтвержденная проецированием и освобождением памяти для файлов \texttt{.so} с помощью \texttt{mmap} и \texttt{munmap}.
\end{enumerate}
Анализ логов \texttt{strace} подтвердил, что все четыре лабораторные работы полностью соответствуют требованиям заданий и корректно реализуют заявленные механизмы межпроцессного взаимодействия и управления ресурсами.