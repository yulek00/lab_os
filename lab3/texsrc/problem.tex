\section*{Условие}

\textbf{Цель работы:} Приобретение практических навыков в:
\begin{itemize}
    \item Освоение принципов работы с файловыми системами
    \item Обеспечение обмена данными между процессами посредством технологии «File mapping»
\end{itemize}

\textbf{Задание:} Составить и отладить программу на языке Си, осуществляющую работу с процессами и взаимодействие между ними в одной из двух операционных систем. В результате работы программа (основной процесс) должен создать для решения задачи один или несколько дочерних процессов. Взаимодействие между процессами осуществляется через системные сигналы/события и/или через отображаемые файлы (memory-mapped files). Необходимо обрабатывать системные ошибки, которые могут возникнуть в результате работы.

\textbf{Вариант:} 3

\textbf{Описание варианта:} Пользователь вводит команды вида: «число число число<конец строки>». Далее эти числа передаются от родительского процесса в дочерний. Дочерний процесс производит деление первого числа на последующие, а результат выводит в файл. Если происходит деление на 0, то тогда дочерний и родительский процесс завершают свою работу. Проверка деления на 0 должна осуществляться на стороне дочернего процесса. Числа имеют тип int. Количество чисел может быть произвольным.

\begin{figure}[h]
    \centering
    \includegraphics[width=\textwidth]{image.png}
    \caption{Схема взаимодействия процессов}
    \label{fig:scheme}
\end{figure}

\section*{Метод решения}

\subsection*{Общее описание алгоритма}

Программа состоит из двух процессов: родительского (parent) и дочернего (child), которые взаимодействуют между собой с использованием:
\begin{itemize}
    \item Общей памяти (shared memory) для передачи данных
    \item Сигналов для синхронизации работы
\end{itemize}

\textbf{Алгоритм работы:}
\begin{enumerate}
    \item Родительский процесс создает общую память и запускает дочерний процесс
    \item Родительский процесс запрашивает у пользователя ввод чисел
    \item Введенные данные записываются в общую память
    \item Родитель посылает сигнал WORK дочернему процессу
    \item Дочерний процесс, получив сигнал, читает данные из общей памяти
    \item Дочерний процесс выполняет деление первого числа на последующие
    \item Если обнаружено деление на ноль, дочерний процесс отправляет SIGTERM родителю и завершается
    \item Результаты записываются в файл
    \item Дочерний процесс отправляет сигнал CONFIRM родителю
    \item Процесс повторяется до ввода команды "exit" или обнаружения деления на ноль
\end{enumerate}

Для реализации использованы системные вызовы POSIX для работы с процессами, сигналами и общей памятью. Данный подход обеспечивает надежное межпроцессное взаимодействие с синхронизацией через сигналы и передачей данных через общую память.