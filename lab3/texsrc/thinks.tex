\section*{Выводы}

В ходе выполнения лабораторной работы я освоила организацию взаимодействия между процессами в Linux с использованием общей памяти и сигналов. Основной задачей было реализовать передачу чисел от родительского процесса дочернему для выполнения операции деления с записью результатов в файл. Для обмена данными была применена технология POSIX shared memory, позволяющая процессам работать с одной областью памяти без копирования данных. Синхронизация выполнялась через пользовательские сигналы SIGUSR1 и SIGUSR2, что обеспечивало корректную последовательность операций. Особое внимание было уделено обработке критических ситуаций, таких как деление на ноль, с реализацией механизма аварийного завершения обоих процессов.