\section{Метод решения}

Алгоритм решения задачи:

\begin{enumerate}
    \item Пользователь в консоль родительского процесса вводит имя файла, которое будет использовано для открытия файла дочерним процессом.
    
    \item Создается объект класса Parent (родительский процесс).
    
    \item Родительский процесс создает канал (pipe) для передачи данных между родительским и дочерним процессами. Затем делает fork для создания дочернего процесса.
    
    \item Дочерний процесс закрывает соответствующий канал для записи, перенаправляет конец канала для чтения на стандартный ввод, закрывает второй конец канала.
    
    \item Дочерний процесс пытается запустить бинарный файл для дочернего процесса, если не получилось — завершается.
    
    \item Родительский процесс закрывает конец канала для чтения и отправляет имя файла для его открытия дочернему процессу.
    
    \item Дочерний процесс пытается открыть указанный файл, если не получилось — процесс завершается.
    
    \item Родительский процесс считывает пользовательский ввод и отправляет дочернему процессу через pipe.
    
    \item Дочерний процесс читает строки чисел, производит деление первого числа на последующие, а результат выводит в файл.
    
    \item Если происходит деление на 0, то дочерний процесс завершает свою работу с ошибкой, после чего родительский процесс также завершает работу.
    
    \item По команде "exit" родитель останавливает дочерний процесс и программа завершается.
\end{enumerate}

\section{Архитектура программы}

\subsection{Структура проекта}

\begin{verbatim}
lab1_os
|-- app
|   |-- childmain.cpp
|   |-- parentmain.cpp
|-- include
|   |-- child.h
|   |-- os.h
|   |-- parent.h
|-- src
|   |-- child.cpp
|   |-- oswin.cpp
|   |-- os.cpp
|   |-- parent.cpp
|-- main.tex
\end{verbatim}

\subsection{Назначение файлов}

\begin{itemize}
    \item \textbf{app/childmain.cpp} — точка входа в программу для дочернего процесса
    \item \textbf{app/parentmain.cpp} — точка входа в программу для родительского процесса
    \item \textbf{include/child.h} — объявление класса Child
    \item \textbf{include/parent.h} — объявление класса Parent
    \item \textbf{include/os.h} — объявление функций управления процессами ОС
    \item \textbf{src/child.cpp} — реализация класса Child
    \item \textbf{src/parent.cpp} — реализация класса Parent
    \item \textbf{src/oswin.cpp} — реализация функций для работы с процессами и каналами в Windows
    \item \textbf{src/os.cpp} — реализация функций для работы с процессами и каналами в Linux
\end{itemize}

\section{Описание программы}

\subsection*{Класс Child}

\subsubsection*{Поля класса}
\begin{itemize}
    \item \texttt{std::string filename} — название файла для открытия
    \item \texttt{FILE* file} — указатель на файл для записи
\end{itemize}

\subsubsection*{Основные методы}
\begin{itemize}
    \item \texttt{Child(const std::string\& filename\_arg)} — конструктор, открывающий файл
    \item \texttt{void ProcessDivision()} — получает из канала строки, выполняет деление и записывает в файл
    \item \texttt{\texttildelow Child()} — деструктор, закрывающий файл
\end{itemize}

\subsubsection*{Вспомогательные методы}
\begin{itemize}
    \item \texttt{std::vector<int> parse\_numbers(const std::string\& input\_line)} — парсит строку с числами
\end{itemize}

\subsection*{Класс Parent}

\subsubsection*{Поля класса}
\begin{itemize}
    \item \texttt{int pipe\_write\_end} — конец канала для записи в дочерний процесс
    \item \texttt{int pipe\_read\_end} — конец канала для чтения
\end{itemize}

\subsubsection*{Основные методы}
\begin{itemize}
    \item \texttt{Parent()} — конструктор
    \item \texttt{void CreateChildProcess(const std::string\& filename)} — создает дочерний процесс
    \item \texttt{void Input()} — получает пользовательский ввод и записывает в канал
    \item \texttt{void EndChild()} — завершает дочерний процесс
    \item \texttt{\texttildelow Parent()} — деструктор
\end{itemize}

\subsection*{Модуль OS}

\subsubsection*{Основные функции}
\begin{itemize}
    \item \texttt{int CreateChildProcess(const StartProcess\& args)} — создает дочерний процесс
    \item \texttt{bool CreatePipe(PipeHandle\& readpipe, PipeHandle\& writepipe)} — создает канал
    \item \texttt{int WritePipe(PipeHandle pipe, const void* buf, int count)} — запись в канал
    \item \texttt{int ReadPipe(PipeHandle pipe, void* buf, int count)} — чтение из канала
    \item \texttt{void ClosePipe(PipeHandle pipe)} — закрытие канала
    \item \texttt{void Exit(int code)} — завершение текущего процесса
    \item \texttt{int WaitForChild()} — ожидание завершения дочернего процесса
\end{itemize}

\section{Результаты}

Программа получает на вход название файла, создает дочерний процесс, который открывает этот файл. Далее все введенные пользователем строки с числами передаются в дочерний процесс, который выполняет деление первого числа на последующие и записывает результаты в файл.

\subsection{Пример работы программы}

\begin{verbatim}
Входные данные: "10 2 5"
Результат в файле: 
10 / 2 = 5
10 / 5 = 2

Входные данные: "100 4 25"
Результат в файле:
100 / 4 = 25
100 / 25 = 4
\end{verbatim}