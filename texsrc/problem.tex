\section{Условие}

\textbf{Цель работы:}
Приобретение практических навыков в:
\begin{itemize}
    \item Управлении процессами в ОС
    \item Обеспечение обмена данных между процессами посредством каналов
\end{itemize}

\textbf{Задание:}
Составить и отладить программу на языке C++, осуществляющую работу с процессами и взаимодействие между ними в операционной системе. В результате работы программа (основной процесс) должен создать для решения задачи один дочерний процесс. Взаимодействие между процессами осуществляется через каналы (pipe). Необходимо обрабатывать системные ошибки, которые могут возникнуть в результате работы.

\textbf{Вариант 3:}
Родительский процесс создает дочерний процесс. Первой строчкой пользователь в консоль родительского процесса пишет имя файла, которое будет передано при создании дочернего процесса. Родительский и дочерний процесс должны быть представлены разными программами. Родительский процесс передает команды пользователя через pipe, который связан со стандартным входным потоком дочернего процесса.

Пользователь вводит команды вида: «число число число». Далее эти числа передаются от родительского процесса в дочерний. Дочерний процесс производит деление первого числа на последующие, а результат выводит в файл. Если происходит деление на 0, то тогда дочерний и родительский процесс завершают свою работу. Проверка деления на 0 должна осуществляться на стороне дочернего процесса. Числа имеют тип int. Количество чисел может быть произвольным.

\begin{figure}[h]
\centering
\includegraphics[width=0.8\textwidth]{process.png}
\caption{Схема работы процессов}
\label{fig:scheme}
\end{figure}