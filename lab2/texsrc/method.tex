\section*{Метод решения}

Алгоритм решения задачи:

\begin{enumerate}
    \item Пользователь запускает программу с двумя параметрами: радиус окружности и максимальное количество потоков.
    \item Программа генерирует 1 000 000 000 (миллиард) случайных точек в квадрате, описанном вокруг окружности заданного радиуса.
    \item Точки равномерно распределяются между потоками.
    \item Каждый поток независимо генерирует свои точки и подсчитывает, сколько из них попадают внутрь окружности.
    \item Для синхронизации доступа к общему счетчику попаданий используется мьютекс.
    \item После завершения работы всех потоков вычисляется площадь окружности по формуле:
\end{enumerate}

\[ S = 4 \times R^2 \times \frac{\text{количество попавших точек}}{\text{общее количество точек}} \]

\textbf{Математическая основа метода Монте-Карло:}

Метод основан на использовании случайных выборок для решения детерминированных задач. Вероятность того, что случайно выбранная точка внутри квадрата со стороной $2R$ попадет в окружность радиуса $R$, равна отношению площади окружности к площади квадрата:

\[ P = \frac{\pi R^2}{(2R)^2} = \frac{\pi}{4} \]

Следовательно, площадь окружности можно оценить как:

\[ S \approx 4 \times R^2 \times \frac{M}{N} \]

где $M$ --- количество точек внутри окружности, $N$ --- общее количество точек.