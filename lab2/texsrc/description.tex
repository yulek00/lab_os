\section*{Описание программы}

\textbf{Архитектура программы:}
\begin{verbatim}
lab2/
├── build/
├── include/
│   ├── exceptions.h
│   └── threads.h
├── src/
│   └── threads.cpp
└── main.cpp
\end{verbatim}

\textbf{Основные компоненты:}

\begin{itemize}
    \item \texttt{main.cpp} --- основная программа, реализующая метод Монте-Карло для вычисления площади окружности.
    \item \texttt{include/exceptions.h} --- объявление классов исключений.
    \item \texttt{include/threads.h} --- объявление класса Thread для работы с потоками.
    \item \texttt{src/threads.cpp} --- реализация методов класса Thread.
\end{itemize}

\textbf{Основные функции и структуры:}

\begin{itemize}
    \item \texttt{struct GlobalResult} --- хранит общие данные для всех потоков (общее количество попаданий и мьютекс для синхронизации).
    \item \texttt{struct ThreadArgs} --- аргументы, передаваемые в каждый поток (радиус, количество точек для генерации, указатель на глобальные данные).
    \item \texttt{void* calculate\_area\_chunk(void* args)} --- функция, выполняемая в каждом потоке. Генерирует случайные точки и подсчитывает попадания внутрь окружности.
    \item \texttt{int main(int argc, char* argv[])} --- точка входа в программу, управляет созданием и синхронизацией потоков.
\end{itemize}

Программа использует мьютекс (\texttt{pthread\_mutex\_t}) для синхронизации доступа к общему счетчику, что предотвращает гонки данных при одновременной записи от нескольких потоков.