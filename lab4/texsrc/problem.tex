\section*{Условие}

\textbf{Цель работы:}
\begin{itemize}
    \item Приобретение практических навыков в:
    \begin{itemize}
        \item Создании динамических библиотек;
        \item Создании программ, использующих функции динамических библиотек.
    \end{itemize}
\end{itemize}

\textbf{Задание:}

Требуется создать динамические библиотеки, которые реализуют заданный вариантом функционал. Далее использовать данные библиотеки двумя способами:
\begin{enumerate}
    \item Во время компиляции (на этапе «линковки» / linking);
    \item Во время исполнения программы. Библиотеки загружаются в память с помощью интерфейса ОС для работы с динамическими библиотеками.
\end{enumerate}

В конечном итоге, в лабораторной работе необходимо получить следующие части:
\begin{itemize}
    \item Динамические библиотеки, реализующие контракты, заданные вариантом;
    \item Тестовая программа (программа №1), которая использует одну из библиотек, используя информацию, полученную на этапе компиляции;
    \item Тестовая программа (программа №2), которая загружает библиотеки, используя только их относительные пути и контракты.
\end{itemize}

Провести анализ двух типов использования библиотек. Пользовательский ввод для обеих программ должен быть организован следующим образом:
\begin{enumerate}
    \item Если пользователь вводит команду «0», то программа завершает работу. В программе №2 команда «3» переключает реализацию (модификация стандартного требования для удобства интерфейса);
    \item «1 arg1 arg2 ... argN», где после «1» идут аргументы для первой функции, предусмотренной контрактами;
    \item «2 arg1 arg2 ... argM», где после «2» идут аргументы для второй функции, предусмотренной контрактами.
\end{enumerate}

\textbf{Вариант: 24}

\begin{center}
\begin{tabularx}{\textwidth}{|c|X|X|X|X|}
\hline
№ & Описание & Сигнатура & Реализация 1 & Реализация 2 \\
\hline
3 & Вычисление наибольшего общего делителя (НОД) для двух натуральных чисел & \texttt{int GCF(int A, int B)} & Алгоритм Евклида & Наивный алгоритм: перебор всех возможных делителей от 1 до $\min(A, B)$ \\
\hline
7 & Вычисление площади фигуры по двум параметрам & \texttt{float Square(float A, float B)} & Площадь прямоугольника: $A \cdot B$ & Площадь прямоугольного треугольника: $0.5 \cdot A \cdot B$ \\
\hline
\end{tabularx}
\end{center}

\section*{Метод решения}

Алгоритм решения задачи:

\begin{enumerate}
    \item Пользователь вводит команду в консоль одной из тестовых программ:
    \begin{itemize}
        \item \texttt{0} — выход из программы (в обеих версиях);
        \item \texttt{3} — переключение реализаций (только для программы №2, \texttt{prog\_dynamic});
        \item \texttt{1 A B} — вызов функции \texttt{GCF(A, B)};
        \item \texttt{2 A B} — вызов функции \texttt{Square(A, B)}.
    \end{itemize}

    \item Для программы \texttt{prog\_static}:
    \begin{itemize}
        \item Связывание с библиотеками \texttt{libgcf\_euclid.so} и \texttt{libsquare\_rect.so} происходит на этапе линковки;
        \item При запуске ОС автоматически загружает указанные \texttt{.so}-файлы;
        \item Вызовы функций разрешаются до начала выполнения основной логики.
    \end{itemize}

    \item Для программы \texttt{prog\_dynamic}:
    \begin{itemize}
        \item На старте загружаются библиотеки \texttt{./libgcf\_euclid.so} и \texttt{./libsquare\_rect.so} с помощью \texttt{dlopen};
        \item Через \texttt{dlsym} получаются указатели на функции \texttt{GCF} и \texttt{Square};
        \item При вводе команды \texttt{3} текущие библиотеки выгружаются (\texttt{dlclose}), загружаются альтернативные (\texttt{libgcf\_naive.so}, \texttt{libsquare\_triangle.so}), обновляются указатели на функции;
        \item Результат выводится в консоль.
    \end{itemize}

    \item Все функции экспортируются с \texttt{extern "C"}, чтобы избежать name mangling и обеспечить корректную работу \texttt{dlsym}.

    \item Память в библиотеках не выделяется динамически, так как обе функции возвращают простые типы (\texttt{int}, \texttt{float}), поэтому освобождение памяти не требуется.

    \item Обработка ошибок:
    \begin{itemize}
        \item При ошибке \texttt{dlopen} или \texttt{dlsym} выводится сообщение и программа завершается с кодом \texttt{EXIT\_FAILURE};
        \item Все ресурсы (дескрипторы библиотек) освобождаются вручную через \texttt{dlclose} при завершении.
    \end{itemize}
\end{enumerate}

\section*{Архитектура программы}

\begin{verbatim}
lab4-var24/
├── CMakeLists.txt
├── include/
│   └── functions.h
├── src/
│   ├── gcf_euclid.cpp
│   ├── gcf_naive.cpp
│   ├── square_rect.cpp
│   └── square_triangle.cpp
├── main_static.cpp
└── main_dynamic.cpp
\end{verbatim}

\section*{Ссылки}

\begin{itemize}
    \item https://pubs.opengroup.org/onlinepubs/9699919799/functions/dlopen.html
    \item https://pubs.opengroup.org/onlinepubs/9699919799/functions/dlsym.html
    \item https://pubs.opengroup.org/onlinepubs/9699919799/functions/dlclose.html
    \item https://man7.org/linux/man-pages/man3/dlopen.3.html
    \item https://www.ibm.com/developerworks/library/l-dynamic-libraries/
\end{itemize}